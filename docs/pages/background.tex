\chapter{图像有损压缩技术的背景}

\section{图像压缩的必要性}

图像数据量庞大,给存储和传输带来了许多困难。
一张尺寸为3840x2180的原始图像,如果每个像素使用32bit来表示(RGBA),那么需要的内存为3840x2180x4 = 33484800 Byte ≈31.9M。相应的,如果拍摄1 min 30 fps这样规格的4k视频,那么需要的存储空间将会达到3840x2180x4x30x60 ≈ 56.1G!

数据压缩的目的就是通过去除这些数据冗余来减少表示数据所需的比特数。由于图像数据量的庞大,在存储、传输、处理时非常困难,因此图像数据的压缩就显得非常重要。
信息时代带来了“信息爆炸”,使数据量大增,因此,无论传输或存储都需要对数据进行有效的压缩。在遥感技术中,各种航天探测器采用压缩编码技术,将获取的巨大信息送回地面。
图像压缩是数据压缩技术在数字图像上的应用,它的目的是减少图像数据中的冗余信息从而用更加高效的格式存储和传输数据。


\section{图像压缩的基本原理}

图像数据之所以能被压缩,就是因为图像数据中存在着冗余部分。
图像数据的冗余主要表现为4种:
\begin{itemize}
    \item \textbf{空间冗余}。一幅图像表面上各像素点之间往往存在着空间连贯性,相邻像素之前也存在相关性,由此产生的空间冗余;
    \item \textbf{时间冗余}。视频的相邻帧往往包含相同的背景和移动物体,相邻帧之间存在相关性,由此产生的时间冗余;
    \item \textbf{频谱冗余}。不同彩色平面或频谱带之间存在相关,由此产生的频谱冗余;
    \item \textbf{视觉冗余}。人类的视觉系统由于受生理特性的限制,对于图像场的注意是非均匀的,人对细微的颜色差异感觉不明显。
\end{itemize}